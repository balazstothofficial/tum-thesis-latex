\chapter{\abstractname}

Functional lists are a handy and easy-to-use data structure in functional programming. However, compared to imperative arrays, |lookup| and |update| operations on lists have a significantly worse runtime of $\mathcal{O}(n)$ instead of $\mathcal{O}(1)$. For this reason, we created an automatic refinement of linearly used lists to arrays and also non-linearly used lists to diff arrays. Diff arrays are persistent and store their updates in a tree-like structure next to a plain array. With them, we can have lookup and update operations in $\mathcal{O}(1)$ for its most recent version.\\
Parallel to the refinement, we will automatically create equivalence proofs using the Isabelle/HOL theorem prover and its Imperative/HOL and separation logic facilities to ensure that the two versions of the program are doing the same. 
