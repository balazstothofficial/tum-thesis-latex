\chapter{Separation Logic}\label{chapter:separation-logic}

When reasoning about programs that mutate memory, we need to keep track of which parts of the memory can get mutated by which part of the program. Separation logic copes with that by representing satisfiability by a heap and having a conjunction operator that states that the heap can be split into two disjoint parts \parencite{Calcagno2007}. Separation logic extends Hoare's logic, such that it is possible to reason about local mutable data structures \parencite{Reynolds}.

\section{Separation Logic in Imperative/HOL}\label{section:separation-logic-hol}

The separation logic implementation of Imperative/HOL \parencite{Separation_Logic_Imperative_HOL-AFP} is formalized following \cite{Calcagno2007} \parencite{Lammich_2017}. Assertions are the clauses of the separation logic and describe if a partial heap\footnote{A heap, which is restricted to a specific set of adresses} satisfies a predciate and a well-formedness condition \parencite[p.482]{Lammich_2017}.\\
The three constant atomic assertions are |true|, |false| and |emp|: All heaps satisfy |true|, no heap satisfies |false| and just the empty heap satisfies |emp| \parencite[p.482]{Lammich_2017}. Moreover, |p|$\rightarrow_r$|v|, |p|$\rightarrow_a$|xs|, and $\uparrow$|b| are the remaining (non-constant) atomic assertions \parencite{Lammich_2017}. |p|$\rightarrow_r$|v| describes a heap, where value |v| is at address |p| \parencite{Lammich_2017}. Analogously, |p|$\rightarrow_a$|a| is a heap where an array |a| holds the same values as the list |xs| and is at the address |p| \parencite{Lammich_2017}. Furthermore, we can model additional conditions using the pure assertion $\uparrow$|b| \parencite{Lammich_2017}. It describes an empty heap if the boolean clause |b| is true and no heap otherwise \parencite{Lammich_2017}.\\
The Imperative/HOL separation logic lifts the standard boolean connectives to assertions, such that they form a boolean algebra themselves \parencite{Lammich_2017}. Doing so, the separation conjunction |P| $*$ |Q| means that the heap consists of two disjoint parts, where one satisfies |P| and the other |Q| \parencite[p.483]{Lammich_2017}. Additionally, there are also has lifted versions of the universal and existential quantifiers and entailment \parencite{Lammich_2017}.\\
The library builds a powerful proof automation for Hoare triples ($\langle$|P|$\rangle$ |c| $\langle$|Q|$\rangle$\footnotemark) with assertions as pre- and postconditions on top of that \parencite{Lammich_2017}. Its main method is called |sep|\_|auto| and combines among other things simplification and application of previously defined separation logic rules \parencite{Lammich_2017}. If the method gets stuck, it shows its current state, such that it is possible to continue the proof manually \parencite{Lammich_2017}. For a more detailed description we refer to \cite{Lammich_2017}.

\footnotetext{Read: "If precondtition $P$ is satisfied and program $c$ runs, then its result fulfills the postcondition $Q$". When reasoning about garbage-collected languages, $\langle P \rangle c \langle Q \rangle_t$ can be used as short form of  $\langle P \rangle c \langle \lambda x. Q x * true \rangle$}
